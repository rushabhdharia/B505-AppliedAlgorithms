\documentclass{article}
\usepackage{amsmath}
\title{Written Assignment 1}
\author{Rushabh Ashok Dharia}
\begin{document}
\maketitle
Q1.
\begin{align*}
f(x) = 3x^2+5x+3 \\
g(x) = 2x^3+x-100 \\
f(x) = O(g(x)) \\
To~{} show~{} that~{} \exists~{} c,n_0~{}	such~{} that~{}\\
0 \leq 3n^2+5n+3 \leq c(2n^3+n-100)~{} \forall n\geq n_0\\ 
Set~{} c=1~{} and~{} n_0 = 5 \\
3(25)+25+3 \leq 1(2(125)+5-100) \\
103\leq155 \\
Q.E.D  \\
Therefore,~{} 0 \leq 3n^2+5n+3 \leq 1(2n^3+n-100)~{} \forall n\geq 5
\end{align*}	 
Q2.
\begin{align*}
The~{} running~{} time~{} of~{} the~{} program~{} given~{} is~{}:
(\frac{n}{2})^2
\end{align*}
\begin{description}
	Consider n = 10\\
	The outer loop will run 5 times\\
	The inner loop will run 5 times\\
	In total the convoluted loop will run 5x5=25 times\\
Even for other values of n we can verify that  
\end{description}
\begin{align*}
The~{} running~{} time~{} of~{} the~{} program~{} will~{} be~{}
(\frac{n}{2})^2\\
Therefore,~{} the~{} program~{} will~{} have~{} its~{} upper~{} bound~{} as~{} O(n^2)
\end{align*}
Q3.
\begin{align*}
To~{} prove~{} that~{}f(n) = \frac{1}{n} = O(1) \\
Therefore,~{}f(n) = \frac{1}{n}~{} and~{}
g(n) = O(1) \\
\\
To~{} show~{} that~{} \exists~{} c,n_0~{}	such~{} that~{}\notag\\
0 \leq \frac{1}{n} \leq c(1)~{} \forall n\geq n_0\\ 
Set~{} c=1~{} and~{} n_0 = 1 \\
\frac{1}{1} \leq 1(1) \\
1\leq1\notag\\
Q.E.D \notag \\
Therefore,~{} 0 \leq \frac{1}{n} \leq 1(1)~{} ~{} \forall n\geq 1
\end{align*}
Q4.
\begin{align*}
Given~{} f(n) = O(g(n))~{} and~{} f(n) = O(h(n)) \\
Example~{} for~{} g(n) = O(h(n))\\
Let~{} f(n) = x~{} and~{} g(n) = x^2~{} and~{} h(n) = x^3 \\
Here,~{} f(n) = O(g(n))~{} i.e.~{} x=O(x^2)~{} and\\
f(n) = O(h(n))~{} i.e.~{} x=O(x^3)\\
Also,~{}x^2 \leq x^3~{} \forall n \geq 0\\
Therefore,~{} x^2=0(x^3)~{} by~{} definition:~{}0 \leq x^2 \leq c(x^3)~{} \forall n \geq 1~{} and~{} c=1\\
Q.E.D\\
\\
Example~{} for~{} g(n) \neq O(h(n)) \\
Let~{} f(n) = x~{} and~{} g(n) = x^8~{} and~{} h(n) = x^3 \\
Here,~{} f(n) = O(g(n))~{} i.e.~{} x=O(x^8)~{} and\\
f(n) = O(h(n))~{} i.e.~{} x=O(x^3)\\
But,~{}x^8 \geq x^3~{} \forall n \geq 0\\
Therefore,~{} x^8 \neq O(x^3)~{} \\
Since,~{} x^3 = O(x^8)~{} by~{} definition:~{}0 \leq x^3 \leq c(x^8)~{} \forall n \geq 1~{} and~{} c=1\\
Therefore,~{} g(n) \neq O(h(n))\\
Since~{} g(n)=\Omega(h(n))\\
Q.E.D\\
\end{align*}	
Q5.
i)
\begin{align*}
f(n) = 2^n \\
g(n) = n^2 \\
f(n) = O(g(n)) \\
To~{} show~{} that~{} \exists~{} c,n_0~{}	such~{} that~{}\\
0 \leq 2^n \leq c(n^2)~{} \forall n\geq n_0\\ 
Set~{} c=1~{} and~{} n_0 = 2 \\
2^2 \leq 1(2^2) \\
4\leq 4 \\
Q.E.D  \\
Therefore,~{} 0 \leq 2^n \leq 1(n^2)~{} \forall n\geq 2\\
\end{align*}
ii)
\begin{align*}
f(n) = 2^n \\
g(n) = 3^n \\
f(n) = O(g(n)) \\
To~{} show~{} that~{} \exists~{} c,n_0~{}	such~{} that~{}\\
0 \leq 2^n \leq c(3^n)~{} \forall n\geq n_0\\ 
Set~{} c=1~{} and~{} n_0 = 1 \\
2^1 \leq 1(3^1) \\
2\leq 3 \\
Q.E.D  \\
Therefore,~{} 0 \leq 2^n \leq 1(3^2)~{} \forall n\geq 1\\
\end{align*}
iii)
\begin{align*}
f(n) = log~{}n \\
g(n) = log^2n \\
f(n) = O(g(n)) \\
To~{} show~{} that~{} \exists~{} c,n_0~{}	such~{} that~{}\\
0 \leq log~{}n \leq c(log^2n)~{} \forall n\geq n_0\\ 
Set~{} c=1~{} and~{} n_0 = 4 \\
log4 \leq 1(log^24) \\
2\leq 2^2 \\
2\leq 4 \\
Q.E.D  \\
Therefore,~{} 0 \leq log~{}n \leq 1(log^n)~{} \forall n\geq 2\\
\end{align*}
iv)
\begin{align*}
f(n) = n^\sqrt{n} \\
g(n) = n^n \\
f(n) = O(g(n)) \\
To~{} show~{} that~{} \exists~{} c,n_0~{}	such~{} that~{}\\
0 \leq n^{\sqrt{n}} \leq c(n^n)~{} \forall n\geq n_0\\ 
Set~{} c=1~{} and~{} n_0 = 1 \\
1^{\sqrt{1}} \leq 1(1^1) \\
1\leq 1 \\
Q.E.D  \\
Therefore,~{} 0 \leq n^{\sqrt{n}} \leq c(n^n)~{} \forall n\geq 1\\
\end{align*}
\end{document}

